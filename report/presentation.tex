\documentclass{beamer}
 
\usepackage[american]{babel}
\usepackage[utf8]{inputenc}
\usepackage[T1]{fontenc}
\usepackage{array}
\usepackage{tikz}
\usepackage{epstopdf}
\usetheme{Copenhagen}
\setbeamertemplate{navigation symbols}{}
\title[Audiovisual recognition of drum sequences]{Audiovisual recognition of drum sequences}
\author[A.~Bérard, C.~Robin]{Alexandre Bérard, Charles Robin}

\date{January 10th, 2014}

\newcommand*\oldmacro{}%
\let\oldmacro\insertshorttitle%
\renewcommand*\insertshorttitle{%
  \oldmacro\hfill%
  \insertframenumber\,/\,\inserttotalframenumber}

\begin{document}

    \begin{frame}
    \titlepage
    \end{frame}

    \section{Introduction}    
    \begin{frame}
        \frametitle{Introduction}
        \emph{ENST-Drums}: \emph{30 Go} of drum audio and video sequences, played by three different drummers on their own drum kit.
        \vspace*{0.5cm}
       
        Four types of sequence: \emph{hits}, \emph{phrases}, \emph{soli}, \emph{accompaniment}. All sequences are annoted, with the time of each stroke and the corresponding instrument.
        \vspace*{0.5cm}

        Possible instruments: snare drum, bass drum, cymbals (chinese ride, crash, splash, etc.), hi-hat, toms (low tom, mid tom, etc.)
    \end{frame}

    \begin{frame}
        \frametitle{Exercise}
        Different possible classification tasks:
        \begin{itemize}
            \item Recognize the drummer;
            \item Recognize the tool used to hit (stick, brush, mallet);
            \item The instrument that is hit (snare drum, bass drum, etc.);
            \item Or a higher-level category of instrument (e.g. membranes versus plates).
        \end{itemize}
        We could use audio features, video features, or both of them.
    \end{frame}
    \begin{frame}
        \frametitle{Our goal}
        We tried three classification tasks: recognize the instrument type within three categories.
        \begin{itemize}
            \item \emph{Super-category}: membrane, plate
            \item \emph{Basic-level}: bass drum, snare drum, tom, cymbal, hi-hat
            \item \emph{Sub-category}: like \emph{basic-level}, but toms are subdivised into low tom, low mid tom, etc., cymbals into splash, ride and crash cymbals.
        \end{itemize}
    \end{frame}
    \begin{frame}
        \frametitle{Bibliography}
        \footnotesize 
        \nocite{*}
        \bibliographystyle{plain}
        \bibliography{presentation}
        \normalsize
    \end{frame}
    \begin{frame}
        \frametitle{Whole process}
        \begin{enumerate}
            \item Data selection
            \item Data segmentation (extraction of strokes out of the sequences)
            \item Features extraction (convert audio segments into vectors)
            \item Normalization of the attributes
            \item Training of a classifier (on the training data)
            \item Evaluation of the classifier
        \end{enumerate}
    \end{frame}
    
    \begin{frame}
        \frametitle{Data segmentation}
        Audio records are sequences of strokes. We must extract those strokes. The first step is to detect the beginning of each stroke.
        This process is called \emph{onset detection}. We use the time defined in the annotations as an oracle.
        \vspace*{0.5cm}
        
        Then, we must define a segment size. It could be either a fixed window (e.g. 200 ms) or the whole audio sequence until the next stroke.
    \end{frame}
    \section{Feature extraction}
    \begin{frame}
        \frametitle{Feature selection}
    \end{frame}
    \begin{frame}
        \frametitle{Chosen features}
        As suggested by Gillet et al. in \cite{Gillet04}, we used the following features:
        \begin{itemize}
            \item Means of 13 MFCC coefficients (starting by $c_0$), using an analysis window of 50 ms and a 50\% overlap.
            \item 4 spectral shape parameters: spectral centroïd, width, skewness and kurtosis; defined as \emph{SpectralShapeStatistics} in Yaafe.
            \item Log-energy in 6 frequency bands (chosen accordingly to the frequency content of each instrument)
        \end{itemize}

    \end{frame}
    \section{Classification}
    \begin{frame}
        \frametitle{Classification}
        Herrera et al. use a k-NN, Gillet et al. prefer a SVM. We tried both of them.
        Evaluation protocol: for each instrument, we measure its precision and recall. We used cross-validation, with 10 folds.

        \begin{center}

3 SVM (C=2, $\sigma$=1, 10 folds)
\begin{tabular}{|c|c|c|c|}
\hline
Instrument&Precision&Recall&F-measure\\
\hline
Bass drum&90.6\%&76.1\%&82.7\%\\
\hline
Snare drum&92.8\%&82.6\%&87.4\%\\
\hline
Hi-hat&84.7\%&94.1\%&89.2\%\\
\hline
Average&89.4\%&84.3\%&86.4\%\\
\hline
\end{tabular}

%SVM (C=2, $\sigma$=1, 10 folds)
%\begin{tabular}{|c|c|c|c|}
%\hline
%Instrument&Precision&Recall&F-measure\\
%\hline
%Bass drum&91.4\%&74.1\%&81.9\%\\
%\hline
%Snare drum&93.0\%&79.6\%&85.8\%\\
%\hline
%Hi-hat&84.7\%&93.2\%&88.8\%\\
%\hline
%Average&89.7\%&82.3\%&85.5\%\\
%\hline
%\end{tabular}
%
%K-NN (k=3, 10 folds)
%\begin{tabular}{|c|c|c|c|}
%\hline
%Instrument&Precision&Recall&F-measure\\
%\hline
%Bass drum&75.3\%&84.7\%&79.7\%\\
%\hline
%Snare drum&87.0\%&83.5\%&85.2\%\\
%\hline
%Hi-hat&83.8\%&89.5\%&86.6\%\\
%\hline
%Average&82.0\%&85.9\%&83.8\%\\
%\hline
%\end{tabular}
%
%Basic-level, SVM (C=2, $\sigma$=1, 10 folds)
%\begin{tabular}{|c|c|c|c|}
%\hline
%Instrument&Precision&Recall&F-measure\\
%\hline
%Cymbal&87.5\%&50.9\%&64.3\%\\
%\hline
%Bass drum&89.5\%&73.4\%&80.7\%\\
%\hline
%Snare drum&90.6\%&81.6\%&85.9\%\\
%\hline
%Tom&90.8\%&33.0\%&48.3\%\\
%\hline
%Hi-hat&79.9\%&89.9\%&84.6\%\\
%\hline
%Average&87.7\%&65.8\%&72.7\%\\
%\hline
%\end{tabular}
%
%Basic-level, 3 SVM (C=2, $\sigma$=1, 10 folds)
%\begin{tabular}{|c|c|c|c|}
%\hline
%Instrument&Precision&Recall&F-measure\\
%\hline
%Cymbal&85.3\%&61.2\%&71.2\%\\
%\hline
%Bass drum&88.3\%&77.5\%&82.5\%\\
%\hline
%Snare drum&91.3\%&82.8\%&86.8\%\\
%\hline
%Tom&88.1\%&40.7\%&55.4\%\\
%\hline
%Hi-hat&85.0\%&87.4\%&86.2\%\\
%\hline
%Average&87.6\%&69.9\%&76.4\%\\
%\hline
%\end{tabular}
%
%Sub-category, SVM (C=2, $\sigma$=1, 10 folds)
%\begin{tabular}{|c|c|c|c|}
%\hline
%Instrument&Precision&Recall&F-measure\\
%\hline
%Splash cymbal&80.0\%&0.0\%&0.0\%\\
%\hline
%Low tom&74.7\%&18.7\%&29.5\%\\
%\hline
%Low mid tom&60.0\%&27.7\%&35.5\%\\
%\hline
%Other cymbal&88.8\%&37.0\%&51.9\%\\
%\hline
%Ride cymbal&83.3\%&42.8\%&56.4\%\\
%\hline
%Closed hi-hat&69.8\%&85.7\%&76.9\%\\
%\hline
%Open hi-hat&82.5\%&61.0\%&70.1\%\\
%\hline
%Mid tom&69.1\%&21.8\%&32.9\%\\
%\hline
%Bass drum&89.8\%&71.6\%&79.6\%\\
%\hline
%Snare drum&88.8\%&83.7\%&86.2\%\\
%\hline
%Lowest tom&97.5\%&25.1\%&37.8\%\\
%\hline
%Crash cymbal&91.7\%&57.5\%&70.4\%\\
%\hline
%Average&81.3\%&44.4\%&52.3\%\\
%\hline
%\end{tabular}

        \end{center}
    \end{frame}

    \section{Conclusion}

    \begin{frame}
        \frametitle{Remarks}
        Good results, because we skipped the \emph{onset detection} step by using an oracle.

        There is room for improvement: feature selection, parameters tuning, etc. We didn't evaluate our system on noisy data.
    \end{frame}
   
    \section{}
    \begin{frame}
        \frametitle{~}

        \begin{center}
            Thank you for listening.
            \underline{Questions?}
        \end{center}
    \end{frame}
\end{document}
